
Ces exercices sont issus de \url{https://adventofcode.com/2015/day/1}.

\subsection{Exercice 1. Année 2015, jour 1}


Vous avez en entrée un fichier texte exo1.txt contenant des parenthèses.
En sortie, vous devrez afficher la valeur d'une variable sur votre terminal de commande.

\subsubsection{première partie}

--- Jour 1 : Pas tout à fait Lisp ---

Le Père Noël essaie de livrer des cadeaux dans un grand immeuble, mais il n'arrive pas à trouver le bon étage - les instructions qu'il a reçues sont un peu confuses. Il commence par le rez-de-chaussée (étage 0) et suit les instructions un caractère à la fois.

Une parenthèse ouvrante, (, signifie qu'il doit monter d'un étage, et une parenthèse fermante, ), signifie qu'il doit descendre d'un étage.

L'immeuble est très haut et le sous-sol est très profond ; il ne trouvera jamais l'étage supérieur ni l'étage inférieur.

Par exemple, (()) et ()()) :

    (()) et ()() donnent tous deux l'étage 0.

    ((( et (()((()( donnent tous deux l'étage 3.
    
    ))((((( donnent également l'étage 3.
    
    ()) et ))() donnent tous deux l'étage -1 (le premier niveau du sous-sol).
    
    ))) et )())()) donnent tous deux l'étage -3.

\textbf{A quel étage les instructions conduisent-elles le Père Noël ?}

\subsubsection{seconde partie}

Maintenant, avec les mêmes instructions, trouvez la position du premier personnage qui le fait entrer dans le sous-sol (étage -1). Le premier caractère dans les instructions a la position 1, le deuxième caractère a la position 2, et ainsi de suite.

Par exemple :

    ) le fait entrer dans le sous-sol à la position du caractère 1.
    
    ()()) le fait entrer dans le sous-sol à la position 5.
    
\textbf{    Quelle est la position du caractère qui fait entrer le Père Noël en premier dans la cave ?
}    

\subsection{Exercice 2. Année 2015, jour 2}

Vous avez en entrée un fichier texte exo2.txt contenant les données $a \times b \times c$ avec a,b et c des nombres.
En sortie, vous devrez afficher la valeur d'une variable sur votre terminal de commande.


\subsubsection{première partie}

--- Jour 2 : On m'a dit qu'il n'y aurait pas de maths...

Les lutins sont à court de papier d'emballage et doivent donc passer une commande pour en obtenir d'autres. Ils ont une liste des dimensions (longueur l, largeur w, et hauteur h) de chaque cadeau, et ne veulent commander que la quantité exacte dont ils ont besoin.

Heureusement, chaque cadeau est une boîte (un prisme rectangulaire droit parfait), ce qui facilite le calcul du papier d'emballage nécessaire pour chaque cadeau : trouver la surface de la boîte, qui est 2*l*w + 2*w*h + 2*h*l. Les lutins ont également besoin d'un peu de papier supplémentaire pour chaque cadeau : la surface du plus petit côté.

C'est l'aire du plus petit côté :

    Un cadeau de dimensions 2x3x4 nécessite 2*6 + 2*12 + 2*8 = 52 mètres carrés de papier d'emballage plus 6 mètres carrés de mou, soit un total de 58 mètres carrés.

    Un cadeau de dimensions 1x1x10 nécessite 2*1 + 2*10 + 2*10 = 42 mètres carrés de papier d'emballage plus 1 mètre carré de jeu, pour un total de 43 mètres carrés.

Tous les nombres de la liste des lutins sont exprimés en pieds. 

\textbf{Combien de mètres carrés de papier d'emballage doivent-ils commander ?}

\subsubsection{seconde partie}


Les lutins sont également à court de ruban. Les rubans étant tous de la même largeur, ils n'ont qu'à se préoccuper de la longueur qu'ils doivent commander, qu'ils voudraient encore une fois exacte.

Le ruban nécessaire pour emballer un cadeau correspond à la distance la plus courte autour de ses côtés, ou au plus petit périmètre d'une face. Chaque cadeau a également besoin d'un nœud fait de ruban ; le nombre de pieds de ruban requis pour un nœud parfait est égal au volume en pieds cubes du cadeau. Ne demandez pas comment ils font le nœud, ils ne vous le diront jamais.

Voici un exemple :

    Un cadeau de dimensions 2x3x4 nécessite 2+2+3+3 = 10 pieds de ruban pour envelopper le cadeau, plus 2*3*4 = 24 pieds de ruban pour le nœud, soit un total de 34 pieds.
    
    Un cadeau aux dimensions 1x1x10 nécessite 1+1+1+1 = 4 pieds de ruban pour envelopper le cadeau plus 1*1*10 = 10 pieds de ruban pour le nœud, soit un total de 14 pieds.

\textbf{Combien de mètres de ruban doivent-ils commander au total ?}


\subsection{Exercice 3. Année 2015, jour 3}


Vous avez en entrée un fichier texte exo3.txt.
En sortie, vous devrez afficher la valeur d'une variable sur votre terminal de commande.

\subsubsection{première partie}

--- Jour 3 : Des maisons parfaitement sphériques dans le vide ---

Le Père Noël livre des cadeaux à une grille bidimensionnelle infinie de maisons.

Il commence par livrer un cadeau à la maison située à son point de départ, puis un elfe du pôle Nord l'appelle par radio pour lui dire où se déplacer ensuite. Les déplacements se font toujours exactement d'une maison vers le nord (\^), le sud (v), l'est (>) ou l'ouest (<). Après chaque déplacement, il livre un autre cadeau à la maison de son nouvel emplacement.

Cependant, le lutin qui se trouve au pôle Nord a bu un peu trop de lait de poule, et ses indications sont donc un peu erronées, de sorte que le Père Noël finit par visiter certaines maisons plus d'une fois. \textbf{Combien de maisons reçoivent au moins un cadeau ?}

Par exemple :

    > livre des cadeaux à 2 maisons : l'une au point de départ, l'autre à l'est.
    
     	\textasciicircum >v< livre des cadeaux à 4 maisons dans un carré, dont deux fois à la maison située à son emplacement de départ/arrivée.
    
     	\textasciicircum v 	\textasciicircum v 	\textasciicircum v 	\textasciicircum v 	\textasciicircum v livre un tas de cadeaux à des enfants très chanceux dans seulement 2 maisons.


\subsubsection{seconde partie}
L'année suivante, pour accélérer le processus, le Père Noël crée une version robotisée de lui-même, Robo-Santa, pour livrer les cadeaux avec lui.

Le Père Noël et Robo-Santa commencent au même endroit (ils livrent deux cadeaux à la même maison de départ), puis se déplacent à tour de rôle en fonction des instructions du lutin, qui lit dans le lait de poule le même scénario que l'année précédente.

\textbf{Cette année, combien de maisons recevront au moins un cadeau ?}

Par exemple :

     	\textasciicircum v livre des cadeaux à 3 maisons, parce que le Père Noël va au nord et que Robo-Santa va au sud.
    
     	\textasciicircum >v< livre maintenant des cadeaux à 3 maisons, et le Père Noël et Robo-Santa reviennent à leur point de départ.
    
     	\textasciicircum v 	\textasciicircum v 	\textasciicircum v 	\textasciicircum v livre maintenant des cadeaux à 11 maisons, le Père Noël allant dans une direction et Robo-Santa dans l'autre.


\subsection{Exercice 4. Année 2015, jour 5}

Vous avez en entrée un fichier texte exo4.txt.
En sortie, vous devrez afficher la valeur d'une variable sur votre terminal de commande.

\subsubsection{première partie}

      --- Jour 5 : Il n'a pas de stagiaires pour cela ? ---

Le Père Noël a besoin d'aide pour déterminer quelles chaînes de son fichier texte sont bonnes ou fausses.

Une chaîne de caractères bonne est une chaîne qui possède toutes les propriétés suivantes :

    Elle contient au moins trois voyelles (aeiou uniquement), comme aei, xazegov, ou aeiouaeiouaeiou.
    
    Elle contient au moins une lettre qui apparaît deux fois de suite, comme xx, abcdde (dd), ou aabbccdd (aa, bb, cc, ou dd).
    Elle ne contient pas les chaînes ab, cd, pq ou xy, même si elles font partie de l'une des autres exigences.

Par exemple :

    ugknbfddgicrmopn est bonne parce qu'elle contient au moins trois voyelles (u...i...o...), une lettre double (...dd...), et aucune des sous-chaînes interdites.
    aaa est bonne parce qu'elle contient au moins trois voyelles et une lettre double, même si les lettres utilisées par les différentes règles se chevauchent.
    
    jchzalrnumimnmhp est fausse car elle n'a pas de lettre double.
    
    haegwjzuvuyypxyu est fausse car elle contient la chaîne xy.
    
    dvszwmarrgswjxmb est fausse car elle ne contient qu'une seule voyelle.

\textbf{Combien de chaînes sont bonnes ?}

\subsubsection{seconde partie}
Conscient de son erreur, le Père Noël a adopté un meilleur modèle pour déterminer si une ficelle est bonne ou fausse. Aucune des anciennes règles ne s'applique, car elles sont toutes clairement ridicules.

Désormais, une chaîne de caractères bonne est une chaîne qui possède toutes les propriétés suivantes :

    Elle contient une paire de deux lettres quelconques qui apparaissent au moins deux fois dans la chaîne sans se chevaucher, comme xyxy (xy) ou aabcdefgaa (aa), mais pas comme aaa (aa, mais elle se chevauche).
    Elle contient au moins une lettre qui se répète avec exactement une lettre entre elles, comme xyx, abcdefeghi (efe), ou même aaa.

Par exemple :

    qjhvhtzxzqqjkmpb est bonne parce qu'elle contient une paire qui apparaît deux fois (qj) et une lettre qui se répète avec exactement une lettre entre elles (zxz).

    xxyxx est bonne parce qu'elle contient une paire qui apparaît deux fois et une lettre qui se répète avec une lettre entre les deux, même si les lettres utilisées par chaque règle se chevauchent.
    
    uurcxstgmygtbstg est fausse car elle a une paire (tg) mais pas de répétition avec une seule lettre entre les deux.
    
    ieodomkazucvgmuy est fausse parce qu'elle contient une lettre répétée avec une lettre entre les deux (odo), mais pas de paire apparaissant deux fois.

\textbf{Combien de chaînes sont bonnes selon ces nouvelles règles ?}
